\chapter{The robot}
The robot used for this application is the LabMate, an old robot that used to have a dated control unit which was repurposed by changing its 
control unit. [reference]

\section{Components}
For a more comprehensive understanding of the components please see [reference].

\subsection{Motor}
Two brushed DC motors operate in parallel to move the robot.

\begin{center}
    \begin{table}[h]
        \begin{tabular} {| m{20em} | m{20em} |}
            \hline
            \textbf{Property} &  \textbf{Value} \\
            \hline
            Voltage & 22V \\
            \hline
            Current & 9A \\
            \hline
            Maximux speed & 1300 rpm \\
            \hline
            Maximux power & 0.13 KW \\
            \hline
            Wheel radius & 75 MM \\
            \hline
            Reduction wheel:motor & 0:10 \\
            \hline
        \end{tabular}
    \end{table}
\end{center}

\subsection{Encoder}
The relative encoder HEDS 9000 is used to get the impulses of each wheel. 
\\ \\
Each encoder has two inputs and two outputs. The inputs are the "plus" and "minus" for the alimentation. The outputs are "Channel A" and "Channel B", the encoder 
uses two channels in order to know if the wheel is going forward or backwards.
\\ \\
The reading of the impulses is performed by an Arduino Due, which uses a library [reference] that performs all the management of channels A and B, the only thing that it returns is the number of impulses.

\subsubsection{Wheel reduction issue}
During the test phase a problem was found in the readings of the encoders, probably because the wheel reduction is not exactly 1:10, 
which leads the encoders to have a 40440 CPR and not a 40000 CPR, which 
should be the correct CPR for the wheel ($CPR_{wheel} = CPR_{encoder} \cdot G_{ratio}$).

\subsection{Arduino}
There are two Arduino boards mounted in the robot. 
\\ \\
One Arduino is responsible for computing the coordinates and sending them over USB.
\\ \\
Another Arduino is responsible for getting over USB the wheels speed and controlling the motor accordingly.
\\ \\
It was chosen to use two separate Arduino boards as one single board couldn't handle the computation operation at the same time as it was controlling the motors.


