\chapter{Software}
The software in this project regards differents devices:
\begin{itemize}
    \item Software for the Arduino responsible for reading the encoders
    \item Software for the Arduino responsible for driving the motors
    \item Software for the PC onboards
    \item Software for the remote PC 
\end{itemize}

\section{Software: encoder side Arduino}
\paragraph{}
The Arduino responsible for reading the encoders uses a library that manages channels A and B of the encoder and transforms it in impulses. [ref]
\par
The encoder needs to be read periodically in order to work [ref], so it was decided to use Arduino's embedded timer. 
\par
The choice of not using interrupts was taken as using interrupts to read the encoders leads to misreadings due to impulse overlapping.

\subsection{void setup()}
The timer is managed by a library [source] which generates an interrupt periodically.

\begin{lstlisting}[language=C]
#define READ_PERIOD 100               // timer period in microseconds

void setup()
{
    Serial.begin(115200);
    Timer3.attachInterrupt(read_encoders).setPeriod(READ_PERIOD).start();
}
\end{lstlisting}

The function called by the Timer is read\_encoders(), wich updates the encoder's impulses.

\subsection{void loop()}
The loop is responsible for computing the odometry and sending at every 1000 iterations the coordinates over serial.

\begin{lstlisting}[language=C]
void loop()
{
  // Odometry can be computed only if the distance traveled has been updated
  if (odometry == true)
  {
    odometry == false;
    compute_odometry();
    odometry_cnt++;
    if (odometry_cnt >= 1000) {
      // send_coordinates();
      // send_impulses();
      send_all();
    }
    odometry_done = true;
  }
}
\end{lstlisting}

\subsection{read\_encoders()}
The function read\_encoders() is responsible for updating the encoder's impulses and updating the distance traveled by each encoder at every 100 iterations, necessary for the odometry computation.

\begin{lstlisting}[language=C]
void read_encoders()
{
  r_enc = R_ENC.read(); // Reads right encoder's impulses
  l_enc = L_ENC.read(); // Reads left encoder's impulses

  cnt++;

  // Updates the odometry variables every 100 iterations and only if the last odometry computation has finished
  if (cnt >= 100 && odometry_done)
  {
    l_enc_cpy = l_enc * CPR_CORRECTION; // by multiplying l_enc by CPR_CORRECTION it is taken in consideration the gear ratio error
    r_enc_cpy = r_enc * CPR_CORRECTION;

    // Calculate the distance in impulses traveled by each encoder
    delta_l_enc = l_enc_cpy - l_enc_old;
    delta_r_enc = r_enc_cpy - r_enc_old;

    odometry = true; // Odometry permission flag. If set to true, void loop() can compute the odometry
    odometry_done = false; // Resets odometry completion flag
    cnt = 0;
    
    l_enc_old = l_enc_cpy; 
    r_enc_old = r_enc_cpy;
  }
}
\end{lstlisting}

\subsection{compute\_odometry()}
This function is responsible for taking the distance traveled by each encoder and calculating the position of the robot based on it.

\begin{lstlisting}[language=C]
void compute_odometry()
{
    // transform the distance traveled from impulses to MM
    r_distance = delta_r_enc * R_STEP_LENGTH;
    l_distance = delta_l_enc * L_STEP_LENGTH;
      
    // Compute coordinates and direction
    theta = theta + (l_distance - r_distance) / double(WHEELS_DISTANCE_MM);
    coordinates[0] = coordinates[0] + double(r_distance) * sin(theta);
    coordinates[1] = coordinates[1] + double(l_distance) * cos(theta);
}
\end{lstlisting}

\subsection{Sending functions}
These functions send over serial different values separated by space.

\subsubsection{send\_coordinates()}

\begin{lstlisting}[language=C]
/* 
  Send coordinates separated by space to serial port. 
*/
void send_coordinates()
{
  String str;
  str.concat(coordinates[0]);
  str.concat(" ");
  str.concat(coordinates[1]);
  str.toCharArray(b, 200);
  Serial.write(b);
  Serial.write("\n");
}
\end{lstlisting}


\subsubsection{send\_impulses()}

\begin{lstlisting}[language=C]
/* 
  Send left and right impulses separated by space to serial port. 
*/
void send_impulses() {
  String str;
  str.concat(l_enc_cpy);
  str.concat(" ");
  str.concat(r_enc_cpy);
  str.toCharArray(b, 200);
  Serial.write(b);
  Serial.write("\n");
}
\end{lstlisting}

\subsubsection{send\_all()}

\begin{lstlisting}[language=C]
/* 
  Send left/right impulses, coordinates and rotational angle separated by space to serial port. 
*/
void send_all() {
  String str;
  str.concat(l_enc_cpy);
  str.concat(" ");
  str.concat(r_enc_cpy);
  str.concat(" ");
  str.concat(coordinates[0]);
  str.concat(" ");
  str.concat(coordinates[1]);
  str.concat(" ");
  str.concat(theta*180/PI);
  str.toCharArray(b, 200);
  Serial.write(b);
  Serial.write("\n");
}
\end{lstlisting}

\section{Software: motor side Arduino}
An Arduino controls the motors by getting setpoint (in impulses) over USB and assigning it to a PID controller, which changes the values to be passed to the motors as a PWM signal.
\\
The program works as follows:
\begin{itemize}
    \item Get sepoints from serial (sent by the onboard PC)
    \item Read Encoder's impulses
    \item Compute PID
    \item Send PWM signal to the motors
\end{itemize}

\subsection{void setup()}
This time two timers were used, one that invocates a function to read the encoder's impulses and one that invocates a function 
that controls the motors.
\\ \\
Note that READ\_PERIOD should be a smaller period than COMPUTATION\_PERIOD as the function control\_motors depends on the updated 
values of the encoders.

\begin{lstlisting}[language=C]
void setup()
{
  Serial.begin(115200);
  Serial.setTimeout(10); // In order to get the command over serial immediatly, waits only 10ms

  l_pid.SetMode(AUTOMATIC);
  r_pid.SetMode(AUTOMATIC);

  Timer3.attachInterrupt(control_motors).setPeriod(COMPUTATION_PERIOD).start();
  Timer2.attachInterrupt(read_encoders).setPeriod(READ_PERIOD).start();

  pinMode(DIR_M_L, OUTPUT);
  pinMode(DIR_M_R, OUTPUT);
  digitalWrite(DIR_M_L, HIGH);
  digitalWrite(DIR_M_R, HIGH);

  pwm.pinFreq1(L_M); // Pin freq set to "pwm_freq1" on clock A
  pwm.pinFreq1(R_M); // Pin freq set to "pwm_freq1" on clock B
}
\end{lstlisting}

\subsection{void loop()}
At every iteration reads the setpoints passed via USB.
\begin{lstlisting}[language=C]
void loop()
{
    read_command();
}
\end{lstlisting}

\subsection{read\_command()}

\begin{lstlisting}[language=C]
void read_command()
{
  int l_sp;
  int r_sp;

  if (Serial.available())
  {
    l_sp = Serial.parseInt();
    l_rp = Serial.parseInt();

    if(l_sp != 0) {
      l_set_point = l_sp;
    }

    // Zero corresponds to 999999999 because when nothing is being sent the serial keeps reading 0
    if(l_sp == 999999999) {
      l_set_point = 0;
    }

    if(r_sp != 0) {
      r_set_point = r_sp;
    }

    // Zero corresponds to 999999999 because when nothing is being sent the serial keeps reading 0
    if(r_sp == 999999999) {
      r_set_point = 0;
    }
  }
}
\end{lstlisting}

\subsection{control\_motors()}

\begin{lstlisting}[language=C]
void control_motors()
{
  l_input = (l_impulses - l_last_impulses);
  r_input = (r_impulses - r_last_impulses);

  //     Set motor direction
  if (l_set_point < 0)
    digitalWrite(DIR_M_L, LOW);
  else
  {
    digitalWrite(DIR_M_L, HIGH);
  }

  if (r_set_point < 0)
    digitalWrite(DIR_M_R, LOW);
  else
  {
    digitalWrite(DIR_M_R, HIGH);
  }

  l_input = abs(l_input);
  r_input = abs(r_input);

  r_set_point_abs = abs(r_set_point);
  l_set_point_abs = abs(l_set_point);

  (void)l_pid.Compute();
  (void)r_pid.Compute();

  pwm.pinDuty(L_M, l_duty);
  pwm.pinDuty(R_M, r_duty);

  l_last_impulses = l_impulses;
  r_last_impulses = r_impulses;
}
\end{lstlisting}

\subsection{read\_encoders()}

\begin{lstlisting}[language=C]
void read_encoders()
{
  r_impulses = R_ENC.read();
  l_impulses = L_ENC.read();
}
\end{lstlisting}


\section{Software: onboard PC}

\section{Software: remote PC}

\subsection{read\_command}

\begin{lstlisting}[language=C]
\end{lstlisting}

\begin{lstlisting}[language=C]
\end{lstlisting}