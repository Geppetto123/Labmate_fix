\chapter{Software}
The software in this project regards differents devices:
\begin{itemize}
    \item Software for the Arduino responsible for reading the encoders
    \item Software for the Arduino responsible for driving the motors
    \item Software for the PC onboards
    \item Software for the remote PC 
\end{itemize}

\section{Software: Encoder side Arduino}
As mentioned before, the Arduino responsible for reading the encoders uses a library that manages channels A and B of the encoder and transforms it in impulses. 
\\ \\
The encoder needs to be read periodically in order to work [reference to encoder library], so it was used a timer. 
\\
Using interrupts to read the encoders leads to impulse loss due to impulse overlapping. 
\\

\subsection{Timer definition}
The timer is managed by a library [source] which generates an interrupt periodically.

\begin{lstlisting}[language=C]
#define READ_PERIOD 100               // timer period in microseconds

void setup()
{
    Serial.begin(115200);
    Timer3.attachInterrupt(read_encoders).setPeriod(READ_PERIOD).start();
}
\end{lstlisting}

The function called by the Timer is read\_encoders(), wich updates the encoder's impulses.

\subsection{Function read\_encoders()}
The function read\_encoders() is responsible for updating the encoder's impulses and updating the distance traveled by each encoder at every 100 iterations, necessary for the odometry computation.

\begin{lstlisting}[language=C]
void read_encoders()
{
  r_enc = R_ENC.read(); // Read the impulses for the right encoder
  l_enc = L_ENC.read(); // Read the impulses for the left encoder

  cnt++;
  // Changes the odometry computation variables every 100 iterations and only if the last odometry computation has finished
  if (cnt >= 100 && odometry_done)
  {
    l_enc_cpy = l_enc * CPR_CORRECTION; // by multiplying l_enc by CPR_CORRECTION it is taken in consideration the gear ratio error
    r_enc_cpy = r_enc * CPR_CORRECTION;

    // Calculate the distance in impulses traveled by each encoder
    delta_l_enc = l_enc_cpy - l_enc_old;
    delta_r_enc = r_enc_cpy - r_enc_old;

    odometry = true;
    odometry_done = false; // Resets odometry completion flag
    cnt = 0;
    
    l_enc_old = l_enc_cpy; 
    r_enc_old = r_enc_cpy;
  }
}
\end{lstlisting}

\subsection{Function compute\_odometry()}
This function is responsible for taking the distance traveled by each encoder and calculating the position of the robot based on it.

\begin{lstlisting}[language=C]
void compute_odometry()
{
    // transform the distance traveled from impulses to MM
    r_distance = delta_r_enc * R_STEP_LENGTH;
    l_distance = delta_l_enc * L_STEP_LENGTH;
      
    // Compute coordinates and direction
    theta = theta + (l_distance - r_distance) / double(WHEELS_DISTANCE_MM);
    coordinates[0] = coordinates[0] + double(r_distance) * sin(theta);
    coordinates[1] = coordinates[1] + double(l_distance) * cos(theta);
}
\end{lstlisting}

\subsection{The loop}
The loop is responsible for computing the odometry and sending at every 1000 iterations the coordinates over serial.

\begin{lstlisting}[language=C]
    void loop()
    {
      if (odometry == true)
      {
        odometry == false;
        compute_odometry();
        delta_l_enc = 0;
        delta_r_enc = 0;
        odometry_cnt++;
        if (odometry_cnt >= 1000) {
          // send_coordinates();
          // send_impulses();
          send_all();
        }
        odometry_done = true;
      }
    }
\end{lstlisting}

\section{Software: Motor side Arduino}

\section{Software: Onboard PC}

\section{Software: Remote PC}