\chapter{Hardware composition}
In this chapter it will be firstly illustrated an overview of the components, so the reader can create an idea of the whole robot. 
Secondly, every component will be examplained in detail. Both sections explaining first the components to be controlled and then the control units. 

\section{Componentry overview}
In this section the components of the LabMate will be exposed in a general way, without much detail.

\subsection{LabMate}
The robot itselft is composed by:
\begin{itemize}
    \item Two brushed motors working in parallel
    \item A wheel for each motor with a reduction of 1:10 (wheel:motor)
    \item Two encoders attached to each motor
\end{itemize}


\subsection{Control unit}
The control unit is composed by:
\begin{itemize}
    \item An Arduino Due for tracking the position
    \item An Arduino Due for driving the wheels
    \item A PC that gets/sends information from/to the Arduinos via USB and gets/sends information from/to a more powerfyl remote PC via SSH
\end{itemize}

The heart of the robot is the PC mounted onboard as it controls and gets information from the Arduinos. The brain of the robot would 
be a remote PC connected to the onboard PC, that would process different types of data and send control information to the onboard PC. 
 The onboard PC could also be used as the brain of the robot, but the whole concept presented here is to process large amounts of data in a more powerful remote PC and use the onboard PC 
to drive the robot.
\\ \\
The Arduino tracking the position gets the distance traveled by each wheel and, based on it, retrieves to the PC the coordinates and the 
direction of the robot via USB.
\\ \\
The Arduino driving the wheels is always listening for a new speed for each wheel (setpoint), sent by the PC over USB.
\\ \\
The choice of using two Arduinos was taken as just one Arduino couldn't handle calculating the coordinates and driving the wheels at the same time.

\section{Components in detail}
In this section all the LabMate components will be exposed in detail.

\subsection{Motors}

\subsubsection{Technical specifications}
\begin{center}
    \begin{table}[h]
        \begin{tabularx}{\textwidth}{|X|X|}
            \hline
            \textbf{Property} &  \textbf{Value} \\
            \hline
            Type & DC/Brushed \\
            \hline
            Voltage & 22V \\
            \hline
            Current & 9A \\
            \hline
            Maximux speed & 1300 rpm \\
            \hline
            Maximux power & 0.13 KW \\
            \hline
            Wheel radius & 75 MM \\
            \hline
            Reduction wheel:motor & 1:10 \\
            \hline
        \end{tabularx}
    \end{table}
\end{center}

\subsection{Encoders}
The encoder is attached to the motor and not the wheel, so it is necessary to take in consideration the wheel reduction to get the correct number of impulses for each rotation.
\\ \\
Channels A and B of the encoders were split in two in order to provide the signals for both Arduinos.

\subsubsection{Gear ratio issue}
During the test phase a problem was found in the readings of the encoders, probably as the wheel reduction is not exactly 1:10, 
which leads the encoders to have a 40440 IPR and not a 40000 IPR, which 
is the theoretical IPR for the wheel ($IPR_{wheel} = 4 \cdot CPR_{encoder} \cdot G_{ratio}$). [reference]

\subsubsection{Technical specifications}
\begin{center}
    \begin{table}[h]
        \begin{tabularx}{\textwidth}{|X|X|}
            \hline
            \textbf{Property} &  \textbf{Value} \\
            \hline
            Maker & Avago Technologies \\
            \hline
            Model & HEDS 9000 \\
            \hline
            Type & Incremental \\
            \hline
            Voltage & 4.5V $\sim$ 5.5V (DC) \\
            \hline
            CPR on motor & 1000 (4000 impulses in quadrature) \\
            \hline
            IPR on wheel & 40000 \\
            \hline
            Real IPR on wheel & $\sim$ 40400 \\
            \hline
            Output & Dual channel/Digital \\
            \hline
        \end{tabularx}
    \end{table}
\end{center}

\textit{IPR indicates "Impulses Per Revolution".}

\subsection{Wheels}
The only information available about the wheels is its radius, which is 75 mm. [ref]
\\ \\
To be correct, it should be mentioned that the robot has other four wheels on its four corners to provide stability and support, nothing more. Also, the two 
driving wheels are attached to a spring to absorb the irregularities present in the field.

\subsection{Arduino Due}
Of all Arduinos tested, the Arduino Due was the only one capable of reading all the impulses generated by the encoders.

\subsubsection{Technical specifications}
\begin{center}
    \begin{table}[h]
        \begin{tabularx}{\textwidth}{|X|X|}
            \hline
            \textbf{Property} &  \textbf{Value} \\
            \hline
            Maker & Arduino \\
            \hline
            Model & Due \\
            \hline
            Microcontroller & AT91SAM3X8E \\
            \hline
            Clock Speed & 84 MHz \\
            \hline
        \end{tabularx}
    \end{table}
\end{center}

\subsubsection{Position tracking Arduino Due}
The Arduino is connected to both encoder's channels A and B for a total of 4 pins used. It is also connected to the onboard PC via USB.
\\ \\
% This Arduino reads the distance percurred by each wheel and transforms it in coordinates, providing also the direction of the robot. The data is retrieved to the 
% onboard PC.

% put images showing the attachements

\subsubsection{Wheels driving Arduino Due}
The Arduino is connected to both encoder's channels A and B. Also, for each wheel the Arduino dedicates two pins: one for the wheel direction and one for the pwm signal. The number of 
pins used by the Arduino in total is 8. It is also connected to the onboard PC via USB.
\\ \\
% This Arduino gets from the onborad PC the desired speeds for both wheels. And based on the impulses read and the setpoints it sends a PWM signal to the motors calculated by a PID type controller.

\subsection{PC}




