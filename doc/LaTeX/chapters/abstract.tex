\chapter*{Abstract}
This paper illustrates the process, functions and tools used in order to achieve a core base for future sensor implementation in laboratory ausiliary robots.
\\ \\
The code reported throughout this paper is made for a specific set of components, it is not a general library, it is meant be used as a \textbf{base} for general purpose robots that share similar characteristics
 with the robot used here or it could be used to replicate the same robot by using the code and the components reported here.
\\ \\
It was decided not to create a library in order to make things as simple as possible both to read and integrate. Previous projects [reference], created general purpose libraries, but any correction and/or modification 
to the code is painful and generate loads of errors.
\\ \\
The robot is able to:
\begin{itemize}
    \item send (remotely) its position (relative to a starting point)
    \item receive (remotely) a twist command where its linear and angular speed are set
\end{itemize}

In order to achieve the results mentioned here it were used odometry calculation...

